% Written on Mon 21 Sep 2020 12:35:52 CEST
% by Jean-Baptiste Caillau, Univ. Cote d'Azur & CNRS/Inria
\documentclass[11pt,a4paper]{article}
\usepackage{hyperref}
\usepackage{amsmath}
\usepackage{mathrsfs}
\usepackage[french]{babel}
\usepackage{graphicx}
\usepackage{tp}
\def\N{\mathbf{N}}
\def\Z{\mathbf{Z}}
\def\Q{\mathbf{Q}}
\def\R{\mathbf{R}}
\def\C{\mathbf{C}}
\def\T{\mathbf{T}}
\def\K{\mathbf{K}}
\def\L{\mathrm{L}}
\def\H{\mathrm{H}}
\def\W{\mathrm{W}}
\def\M{\mathrm{M}}
\def\O{\mathrm{O}}
\def\Im{\mathrm{Im}}
\def\Vect{\mathrm{Vect}}
\def\Min{\mathrm{Min}}
\def\BV{\mathrm{BV}}
\def\Isom{\mathrm{Isom}}
\def\iy{\infty}
\def\d{\mathrm{d}}
\def\t{\ \!^t\!}
\def\tr{\mathrm{tr}}
\def\veps{\varepsilon}
\def\vphi{\varphi}
\def\la{\langle}
\def\ra{\rangle}
\def\noi{\noindent}
\def\cf{\emph{cf.}}
\def\ie{\emph{i.e.}}
\def\etc{\emph{etc.}}
\renewcommand{\tilde}{\widetilde}
\renewcommand{\hat}{\widehat}
\theoremstyle{plain}
\newtheorem{thrm}{Th\'eor\`eme}[section]
\newtheorem{prpstn}{Proposition}[section]
\newtheorem{lmm}{Lemme}[section]
\newtheorem{crllr}{Corollaire}[section]
\newtheorem{dfntn}{D\'efinition}[section]
\theoremstyle{definition}
\newtheorem{rmrk}{Remarque}[section]

\title{Commande optimale}
\shorttitle{Plan du cours}
\numero{Plan du cours}
\date{2020--2021}
\discipline{Commande optimale}
\promotion{Polytech Nice-Sophia --- MAM5-INUM}

\begin{document}
\maketitle

\section*{I. Contr\^ole optimal des EDO}

\paragraph{1. Position du probl\`eme}
\begin{itemize}
  \item exemple en temps minimum
  \item co\^ut de Lagrange
  \item co\^uts de Mayer et Bolza
\end{itemize}

\paragraph{2. Principe du maximum de Pontrjagin}
\begin{itemize}
  \item \'enonc\'e du r\'esultat
  \item conditions de transversalit\'e
  \item exemples
\end{itemize}

\paragraph{3. Preuve du PMP}
\begin{itemize}
  \item PMP faible
  \item constance du hamiltonien le long de l'extr\'emale
\end{itemize}

\paragraph{4. Cas lin\'eaire quadratique}

\begin{itemize}
  \item position du probl\`eme
  \item caract\'erisation de la solution
  \item \'equation de Ricatti
  \item tracking
\end{itemize}

\section*{II. Introduction au contr\^ole des EDP} 

\paragraph{1. Rappels sur les espaces de Hilbert} 

\begin{itemize}
  \item th\'eor\`emes de la projection, de Riesz
  \item th\'eor\`emes de Lions-Stampacchia et Lax-Milgram
\end{itemize}

\paragraph{2. EDP elliptiques (1/2)~: contr\^ole distribu\'e} 

\begin{itemize}
  \item formulation variationnelle de l'EDP
  \item existence et unicit\'e de solution
  \item caract\'erisation de la solution et syst\`eme adjoint
  \item approximation num\'erique
\end{itemize}

\paragraph{3. EDP elliptiques (2/2)~: contr\^ole fronti\`ere} 

\begin{itemize}
  \item formulation variationnelle de l'EDP
  \item existence et unicit\'e de solution
  \item caract\'erisation de la solution et syst\`eme adjoint
  \item approximation num\'erique
\end{itemize}

\section*{III. Apprentissage par renforcement}

\paragraph{1. Model based reinforcement learning} 
\begin{itemize}
  \item contr\^ole optimal stochastique discret
  \item approximation de la dynamique
\end{itemize}

\paragraph{2. Approximate dynamics programming} 
\begin{itemize}
  \item notion de $Q$-fonction
  \item principe de Bellman, $Q$-learning 
  \item lien avec le contr\^ole en temps continu dans le cas LQR
\end{itemize}

\paragraph{3. Direct policy search} 
\begin{itemize}
  \item politiques al\'eatoires
  \item approche "sample to optimize"
\end{itemize}

\paragraph{Organisation et intervenant}

\begin{itemize}
  \item 9 s\'eances de 3H
  \item J.-B.~Caillau (\verb+jean-baptiste.caillau@univ-cotedazur.fr+)
\end{itemize}

\paragraph{\'Evaluation}
\begin{itemize}
  \item 1 EX partiel (coeff.~1)
  \item 1 EX terminal (coeff.~1)
\end{itemize}

\paragraph{Bibliographie}
\begin{enumerate}
%\item Agrachev, A.; Sachkov, Y.
%\emph{Control theory from the geometric viewpoint},
%Encyclop\ae dia of Mathematical Sciences \textbf{87}, Springer, 2004.
\item Blum, J.
\emph{Commande optimale}, Notes de cours Polytech Nice-Sophia, 2016.
\item Evans, L.~C.
\emph{An introduction to mathematical optimal control theory}, Univ. California, 2008.
\item Fleming, W.~H.; Rishel, R.~W.
\emph{Deterministic and stochastic optimal control}, Springer, 1975.
\item Gardner, M.
\emph{The unexpected hanging and other mathematical diversions},
University of Chicago Press, 1991.
\item Recht, B.
A tour of reinforcement learning: the view from continuous control.
arXiv:1806.09460, 2018.
\item Sutton, R.~S.; Barto, A.~G.
\emph{Reinforcement learning: an introduction}, MIT press, 2018.
\item Tr\'elat, E.
\emph{Contr\^ole optimal, th\'eorie et applications}, Vuibert, 2005.

\end{enumerate}

\vfill \begin{flushright}{\footnotesize \emph{En ligne sous}
\texttt{caillau.perso.math.cnrs.fr/commande-mam5}} \end{flushright}
\end{document}
