% Written on Mon 10 Feb 2020 06:46:39 CET
% by Jean-Baptiste Caillau, Universite Cote d'Azur, CNRS, Inria, LJAD
\documentclass[11pt,a4paper]{article}
\usepackage{hyperref}
\usepackage{amsmath}
\usepackage{mathrsfs}
\usepackage[french]{babel}
\usepackage{graphicx}
\usepackage{wasysym}
\usepackage{tp}
\usepackage{version}
\def\N{\mathbf{N}}
\def\Z{\mathbf{Z}}
\def\Q{\mathbf{Q}}
\def\R{\mathbf{R}}
\def\C{\mathbf{C}}
\def\T{\mathbf{T}}
\def\K{\mathbf{K}}
\def\L{\mathrm{L}}
\def\H{\mathrm{H}}
\def\W{\mathrm{W}}
\def\M{\mathrm{M}}
\def\O{\mathrm{O}}
\def\CC{\mathscr{C}}
\def\Im{\mathrm{Im}}
\def\Vect{\mathrm{Vect}}
\def\Min{\mathrm{Min}}
\def\BV{\mathrm{BV}}
\def\Isom{\mathrm{Isom}}
\def\iy{\infty}
\def\d{\mathrm{d}}
\def\t{\ \!^t\!}
\def\tr{\mathrm{tr}}
\def\veps{\varepsilon}
\def\vphi{\varphi}
\def\la{\langle}
\def\ra{\rangle}
\def\noi{\noindent}
\def\cf{\emph{cf.}}
\def\ie{\emph{i.e.}}
\def\etc{\emph{etc.}}
\newcommand{\frp}[2]{\frac{\partial #1}{\partial #2}}
\newcommand{\frpp}[2]{{\partial #1}/{\partial #2}}
\renewcommand{\tilde}{\widetilde}
\renewcommand{\hat}{\widehat}
\theoremstyle{plain}
\newtheorem{thrm}{Th\'eor\`eme}[section]
\newtheorem{prpstn}{Proposition}[section]
\newtheorem{lmm}{Lemme}[section]
\newtheorem{crllr}{Corollaire}[section]
\newtheorem{dfntn}{D\'efinition}[section]
\theoremstyle{definition}
\newtheorem{rmrk}{Remarque}[section]

%\excludeversion{corr}
\includeversion{corr}

\title{Examen}
\shorttitle{Examen}
\numero{Examen}
\date{2019--2020}
\discipline{Commande optimale}
\promotion{Polytech Nice-Sophia --- MAM5-INUM}

\begin{document}
\maketitle

{\bf Dur\'ee 2H. Tous les exercices sont ind\'ependants.
Le bar\`eme pr\'e\-vi\-sion\-nel est indiqu\'e pour chaque exercice.
%Rendre sur des copies s\'epar\'ees l'exercice 1 d'une part, les exercices 2 et 3
%d'autre part.
Documents autoris\'es.}%~: une feuille de notes de cours recto-verso manuscrite.}

% Exercice 1
\begin{Exercice}[8 points] On consid\`ere le probl\`eme \`a temps final $t_f>0$ fix\'e
\[ -q^2(t_f)+\int_0^{t_f} u^2(t)\,\d t \to \min \]
pour la dynamique
\[ \ddot{q}(t)=u(t),\quad t \in [0,t_f], \]
o\`u $q(t)$ et $u(t)$ sont dans $\R$, et o\`u $q(0)=\dot{q}(0)=0$ sont
fix\'es. On laisse $q(t_f)$ et $\dot{q}(t_f)$ libres.
\begin{Question} Mettre la dynamique sous la forme $\dot{x}(t)=f(x(t),u(t))$ avec $f$
que l'on pr\'ecisera.
\end{Question}
\begin{corr} $\RHD$ $f(x,u)=(x_2,u)$
\end{corr}

\begin{Question} Mettre le co\^ut sous forme de Lagrange avec $f^0$ que l'on
pr\'ecisera. [Nota bene~: $(\d /\d t)(q^2)=2q\dot{q}$]
\end{Question}
\begin{corr} $\RHD$ $f^0(x,u)=-2x_1x_2+u^2$ 
\end{corr}

\begin{Question} Donner le hamiltonien du probl\`eme. (En l'absence de contrainte
terminale, on pourra poser $p^0=-1/2$.)
\end{Question}
\begin{corr} $\RHD$ $H(x,p,u)=x_1x_2-u^2/2+p_1x_2+p_2u$
\end{corr}

\begin{Question} D\'eterminer le syst\`eme adjoint.
\end{Question}
\begin{corr} $\RHD$ $\dot{p}_1=-x_2$, $\dot{p}_2=-x_1-p_1$
\end{corr}

\begin{Question} \'Ecrire les conditions de transversalit\'e. 
\end{Question}
\begin{corr} $\RHD$ $p_1(t_f)=p_2(t_f)=0$
\end{corr}

\begin{Question} D\'eterminer le contr\^ole en fonction de l'\'etat et de l'\'etat
adjoint \`a l'aide de la condition de maximisation.
\end{Question}
\begin{corr} $\RHD$ $u(t)=p_2(t)$
\end{corr}

\begin{Question} En d\'eduire le contr\^ole optimal.
\end{Question}
\begin{corr} $\RHD$ On voit que $\ddot{p}_2=-\dot{x}_1-\dot{p}_1=-x_2+x_2=0$, d'o\`u
l'on d\'eduit que $p_2$, et donc $u$, sont des fonctions affines. Comme $p_2(t_f)=0$, il
existe $a \in \R$ tel que $u(t)=a(t_f-t)$.
\end{corr}

\begin{Question} Soit $a \in \R$ une constante, calculer le co\^ut associ\'e au
contr\^ole $u(t)=a(t_f-t)$. Que peut-on en conclure~?
\end{Question}
\begin{corr} $\RHD$ $q(t_f)=at_f^3/3$, que l'on peut rendre arbitrairement petit en
choisissant $a$ n\'egatif. On en d\'eduit que le probl\`eme ne poss\`ede pas de
solution.
\end{corr}

\end{Exercice}

% Exercice 2
\begin{Exercice}[6 points] On consid\`ere le probl\`eme \`a temps final $t_f>0$ fix\'e
\[ \int_0^{t_f} (x_2^2(t)+u_1^2(t)+u_2^2(t))\,\d t \to \min \]
pour la dynamique
\[ \dot{x}_1(t)=-x_2(t)+u_2(t),\quad
   \dot{x}_2(t)= x_1(t)-u_1(t),\quad t \in [0,t_f], \]
o\`u $x(t)$ et $u(t)$ sont dans $\R^2$, et o\`u $x(0)=x_0$ est fix\'e. On laisse
$x(t_f)$ libre.
\begin{Question} Mettre le probl\`eme sous la forme
\[ \int_0^{t_f} \left[ (Cx(t)|x(t))+(Du(t)|u(t)) \right]\,\d t \to \min, \]
\[ \dot{x}(t) = Ax(t)+Bu(t), \]
avec $A$, $B$, $C$ et $D$ que l'on pr\'ecisera. 
\end{Question}
\begin{corr} $\RHD$
\[ A(t) = \left[ \begin{array}{rr}
  0 & -1\\
  1 & 0 \end{array} \right],\quad
   B(t) = \left[ \begin{array}{rr}
  0 & 1\\
 -1 & 0 \end{array} \right], \]
\[ C(t) = \left[ \begin{array}{rr}
  0 & 0\\
  0 & 1 \end{array} \right],\quad
   D(t) = \left[ \begin{array}{rr}
  1 & 0\\
  0 & 1 \end{array} \right]. \]
\end{corr}

\begin{Question} Donner le hamiltonien du probl\`eme. (En l'absence de contrainte
terminale, on pourra poser $p^0=-1/2$.)
\end{Question}
\begin{corr} $\RHD$ $H(x,p,u)=(-1/2)(x_2^2+u_1^2+u_2^2)+p_1(-x_2+u_2)+p_2(x_1-u_1)$
\end{corr}

\begin{Question} D\'eterminer le syst\`eme adjoint.
\end{Question}
\begin{corr} $\RHD$ $\dot{p}_1=-p_2$, $\dot{p}_2=x_2+p_1$
\end{corr}

\begin{Question} \'Ecrire les conditions de transversalit\'e. 
\end{Question}
\begin{corr} $\RHD$ $p_1(t_f)=p_2(t_f)=0$
\end{corr}

\begin{Question} D\'eterminer le contr\^ole en fonction de l'\'etat et de l'\'etat
adjoint \`a l'aide de la condition de maximisation.
\end{Question}
\begin{corr} $\RHD$ $u(t)=(-p_2(t),p_1(t))$
\end{corr}

\begin{Question} On sait que le contr\^ole optimal s'\'ecrit sous la
forme $u(t)=K(t)x(t)$ o\`u $K(t)$ est une matrice qui s'exprime en fonction de la
solution d'une \'equation de Riccati~: expliciter cette \'equation de Riccati.
[On ne demande pas de la r\'esoudre.] Que devient l'expression du contr\^ole optimal
quand $t_f \to \iy$~?
\end{Question}
\begin{corr} $\RHD$ $K(t)=D^{-1}(t)\t B(t)R(t)$ avec
\[ \dot{R}(t)=C-\t AR(t)-R(t)A-R(t)BD^{-1}\t BR(t),\quad
   R(t_f)=0. \]
Quand $t_f \to \iy$, le contr\^ole tend vers le contr\^ole \`a horizon infini dont
l'expression est $u(t)=Kx(t)$ $R$ o\`u $K=D^{-1}\t BR$ et o\`u $R$ est
solution de l'\'equation de Riccati alg\'ebrique (stationnarit\'e)
\[ 0 = C-\t AR-RA-RBD^{-1}\t BR, \]
soit ici $0=C+AR-RA-R^2$.
\end{corr}

\end{Exercice}

% Exercice 3
\begin{Exercice}[6 points]

\begin{Question} Dans le code \verb+bsbfun.m+, entourer la ou les lignes
o\`u sont calcul\'ees les dur\'ees de chaque arc bang.
\end{Question}
\begin{corr} $\RHD$ Voir code joint.
\end{corr}

\begin{Question} Dans le code \verb+bsbfun.m+, la dur\'ee de l'arc singulier est
calcul\'ee selon
\begin{verbatim}
s = abs(z2-z1) / abs(exp(1i*ths)+w);
\end{verbatim}
Justifier ce calcul.
\end{Question}
\begin{corr} $\RHD$ Le long de l'arc singulier, le contr\^ole est nul,
$\theta(t)=\theta_s$ est constant donc, en int\'egrant,
\[ z_2=z_1+s\cdot e^{i\theta_s} \]
o\`u $s$ est la dur\'ee de l'arc.
\end{corr}

\begin{Question} Dans le code \verb+game.m+, entourer la ou les expressions qui
d\'etectent l'absence de coup possible pour la machine.
\end{Question}
\begin{corr} $\RHD$ Voir code joint.
\end{corr}

\begin{Question} On consid\`ere une partie d'Hexapawn pendant laquelle la machine
vient de jouer le coup ci-dessous~:
\begin{verbatim}
[ 0 2 0         [ 0 0 0 
  2 1 1   ---->   2 1 2
  0 0 0 ]         0 0 0 ]
\end{verbatim}
La liste de coups de la machine associ\'ee \`a l'\'etat pr\'ec\'edent (avant son
dernier coup) est 
\begin{verbatim}
{ [ 2 1   , [ 1 2
    3 1 ]     2 3 ] }
\end{verbatim}
Comment cette liste doit-elle \^etre mise \`a jour par renforcement~?
\end{Question}
\begin{corr} La machine perd imm\'ediatement puisque le seul contr\^ole admissible
pour son adversaire, 
\begin{verbatim}
[ 2 2
  1 2 ]
\end{verbatim}
le fait gagner. Le dernier coup jou\'e par la machine est donc supprim\'e par le
renforcement de sorte que la liste de coups devient
\begin{verbatim}
{ [ 2 1
    3 1 ] }
\end{verbatim}
qui, de fait, conduit \`a une victoire de la machine.
\end{corr}

\end{Exercice}

\end{document}

\begin{Exercice}
\begin{Question}
\end{Question}
\begin{corr}
\end{corr}
\end{Exercice}
%\end{Exercice} \vspace*{1em}
